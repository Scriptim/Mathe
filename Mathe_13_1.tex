\part{13/1}

\section{Wiederholung}
\begin{itemize}
  \item relative Häufigkeit: Aussage über ein schon durchgeführtes Zufallsexperiment
  \item Wahrscheinlichkeit: Aussage über zukünftige Zufallsversuche
\end{itemize}
\begin{exercise}{456/5}
  \begin{gather*}
    F = \{(1, 1), (2, 1), (3, 1), (4, 1)\}
  \end{gather*}
  \item [a]
  \begin{gather*}
    E = \{(1, 1), (1, 2), (2, 1)\} \\\\
    F \colon \text{Die zweite Kugel trägt eine $1$} \\
    E \Leftrightarrow \text{Die Summe der Zahlen ist größer als $3$} \\\\
    P(E) = (\frac{2}{6} \cdot \frac{2}{6}) + (\frac{2}{6} \cdot \frac{1}{6}) + (\frac{1}{6} \cdot \frac{2}{6}) = \frac{2}{9} = 0.\overline{2} \\
    P(F) = (\frac{2}{6} \cdot \frac{2}{6}) + (\frac{1}{6} \cdot \frac{2}{6}) + (\frac{2}{6} \cdot \frac{2}{6}) + (\frac{1}{6} \cdot \frac{2}{6}) = P(1) = \frac{2}{6} = \frac{1}{3} = 0.\overline{3}
  \end{gather*}
  \item [b]
  \begin{gather*}
    E \cap F = \{(1, 1), (2, 1)\} \\
    P(E \cap F) = (\frac{2}{6} \cdot \frac{2}{6}) + (\frac{1}{6} \cdot \frac{2}{6}) = \frac{1}{6} = 0.1\overline{6}
  \end{gather*}
  \item [c]
  \begin{gather*}
    E \cup F = \{(1, 1), (1, 2), (2, 1), (3, 1), (4, 1)\} \\
    P(E \cup F) = (\frac{2}{6} \cdot \frac{2}{6}) + (\frac{2}{6} \cdot \frac{1}{6}) + (\frac{1}{6} \cdot \frac{2}{6}) + (\frac{2}{6} \cdot \frac{2}{6}) + (\frac{1}{6} \cdot \frac{2}{6}) = \frac{7}{18} = 0.3\overline{8}
  \end{gather*}
\end{exercise}
\begin{exercise}{456/7}
  \item [a]
  \begin{gather*}
    \text{Gegenwahrscheinlichkeit (keine Sechs werfen): } P(\neg E) = (\frac{5}{6})^5 \approx 0.40 \\
    P(E) = 1 - P(\neg E) = 1 - (\frac{5}{6})^5 \approx 0.60
  \end{gather*}
  \item [b]
  \begin{gather*}
    P = \frac{6}{6} \cdot \frac{5}{6} \cdot \frac{4}{6} \cdot \frac{3}{6} \cdot \frac{2}{6} = \approx 0.09
  \end{gather*}
  \item [c]
  \begin{gather*}
    P = (\frac{5}{6})^4 \cdot \frac{1}{6} \approx 0.08
  \end{gather*}
\end{exercise}
\begin{exercise}{456/9}
  \item [a]
  \begin{gather*}
    P = \frac{1}{13983816} \approx 7.15 \cdot 10^{-8}
  \end{gather*} \\\\
  Bemerkung: Ein Zufallsversuch heißt Laplace-Versuch, wenn alle Ergebnisse gleich wahrscheinlich sind.
  \begin{gather*}
    S = \{e_1, e_2, ..., e_n\} \qquad |S| = n \qquad P(e_i) = \frac{1}{n} \\
    E = \{e_1, ...\} \qquad |E| = k \qquad P(E) = \frac{k}{n}
  \end{gather*}
  $\Rightarrow$ Gleichverteilung
  \item [b]
  \begin{gather*}
    E = \{(1,2,3,4,5,6), (2,3,4,5,6,7), ..., (44,45,46,47,48,49)\} \\
    |E| = 44 \\
    P(E) = \frac{|E|}{|S|} = \frac{44}{13983816} \approx 3.15 \cdot 10^{-6}
  \end{gather*}
  \item [c]
  \begin{gather*}
    P = \frac{43}{49} \cdot \frac{42}{48} \cdot \frac{41}{47} \cdot \frac{40}{46} \cdot \frac{39}{45} \cdot \frac{38}{44} \approx 0.436
  \end{gather*}
\end{exercise}
\begin{exercise}{456/10}
  \item [a]
  \begin{gather*}
    \text{Anteil nach $10$ Tagen: } 0.85^{10} \approx 0.197 \\
    0.85^t = 0.5 \quad\Rightarrow\quad t = log_{0.85}(0.5) \approx 4.265
  \end{gather*}
  \item [b]
  \begin{gather*}
    (1 - p)^8 = 0.5 \quad\Rightarrow\quad p = 1 - \sqrt[8]{(0.5)} \approx 0.083
  \end{gather*}
\end{exercise}
\section{Abzählverfahren}
Bis auf Weiteres gilt: Die Zufallsversuche sind Laplace-Versuche. \\
Anzahl der Ergebnisse $N \quad\Rightarrow\quad P = \frac{1}{N} \quad\text{(Ergebniswahrscheinlichkeit)}$ \\\\
Ziehe $k$ mal aus einer Urne mit $n$ unterscheidbaren Kugeln:
\begin{itemize}
  \item mit Zurücklegen unter Beachtung der Reihenfolge
  \begin{gather*}
    N = n^k
  \end{gather*}
  \item ohne Zurücklegen unter Beachtung der Reihenfolge
  \begin{gather*}
    N = n \cdot (n - 1) \cdot (n - 2) \cdot ... \cdot \underbrace{(n - k + 1)}_{\mathclap{\text{übrige Kugeln vor dem $k$-ten Zug}}} = \frac{n!}{(n - k)!}
  \end{gather*}
\end{itemize}

\part{13/1}

\section{Wiederholung}
\begin{itemize}
  \item relative Häufigkeit: Aussage über ein schon durchgeführtes Zufallsexperiment
  \item Wahrscheinlichkeit: Aussage über zukünftige Zufallsversuche
\end{itemize}
\begin{exercise}{456/5}
  \begin{gather*}
    F = \{(1, 1), (2, 1), (3, 1), (4, 1)\}
  \end{gather*}
  \item [a]
  \begin{gather*}
    E = \{(1, 1), (1, 2), (2, 1)\} \\\\
    F \colon \text{Die zweite Kugel trägt eine $1$} \\
    E \Leftrightarrow \text{Die Summe der Zahlen ist größer als $3$} \\\\
    P(E) = (\frac{2}{6} \cdot \frac{2}{6}) + (\frac{2}{6} \cdot \frac{1}{6}) + (\frac{1}{6} \cdot \frac{2}{6}) = \frac{2}{9} = 0.\overline{2} \\
    P(F) = (\frac{2}{6} \cdot \frac{2}{6}) + (\frac{1}{6} \cdot \frac{2}{6}) + (\frac{2}{6} \cdot \frac{2}{6}) + (\frac{1}{6} \cdot \frac{2}{6}) = P(1) = \frac{2}{6} = \frac{1}{3} = 0.\overline{3}
  \end{gather*}
  \item [b]
  \begin{gather*}
    E \cap F = \{(1, 1), (2, 1)\} \\
    P(E \cap F) = (\frac{2}{6} \cdot \frac{2}{6}) + (\frac{1}{6} \cdot \frac{2}{6}) = \frac{1}{6} = 0.1\overline{6}
  \end{gather*}
  \item [c]
  \begin{gather*}
    E \cup F = \{(1, 1), (1, 2), (2, 1), (3, 1), (4, 1)\} \\
    P(E \cup F) = (\frac{2}{6} \cdot \frac{2}{6}) + (\frac{2}{6} \cdot \frac{1}{6}) + (\frac{1}{6} \cdot \frac{2}{6}) + (\frac{2}{6} \cdot \frac{2}{6}) + (\frac{1}{6} \cdot \frac{2}{6}) = \frac{7}{18} = 0.3\overline{8}
  \end{gather*}
\end{exercise}

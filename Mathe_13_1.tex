\part{13/1}

\section{Wiederholung}
\begin{itemize}
  \item relative Häufigkeit: Aussage über ein schon durchgeführtes Zufallsexperiment
  \item Wahrscheinlichkeit: Aussage über zukünftige Zufallsversuche
\end{itemize}
\begin{exercise}{456/5}
  \begin{gather*}
    F = \{(1, 1), (2, 1), (3, 1), (4, 1)\}
  \end{gather*}
  \item [a]
  \begin{gather*}
    E = \{(1, 1), (1, 2), (2, 1)\} \\\\
    F \colon \text{Die zweite Kugel trägt eine $1$} \\
    E \Leftrightarrow \text{Die Summe der Zahlen ist größer als $3$} \\\\
    P(E) = (\frac{2}{6} \cdot \frac{2}{6}) + (\frac{2}{6} \cdot \frac{1}{6}) + (\frac{1}{6} \cdot \frac{2}{6}) = \frac{2}{9} = 0.\overline{2} \\
    P(F) = (\frac{2}{6} \cdot \frac{2}{6}) + (\frac{1}{6} \cdot \frac{2}{6}) + (\frac{2}{6} \cdot \frac{2}{6}) + (\frac{1}{6} \cdot \frac{2}{6}) = P(1) = \frac{2}{6} = \frac{1}{3} = 0.\overline{3}
  \end{gather*}
  \item [b]
  \begin{gather*}
    E \cap F = \{(1, 1), (2, 1)\} \\
    P(E \cap F) = (\frac{2}{6} \cdot \frac{2}{6}) + (\frac{1}{6} \cdot \frac{2}{6}) = \frac{1}{6} = 0.1\overline{6}
  \end{gather*}
  \item [c]
  \begin{gather*}
    E \cup F = \{(1, 1), (1, 2), (2, 1), (3, 1), (4, 1)\} \\
    P(E \cup F) = (\frac{2}{6} \cdot \frac{2}{6}) + (\frac{2}{6} \cdot \frac{1}{6}) + (\frac{1}{6} \cdot \frac{2}{6}) + (\frac{2}{6} \cdot \frac{2}{6}) + (\frac{1}{6} \cdot \frac{2}{6}) = \frac{7}{18} = 0.3\overline{8}
  \end{gather*}
\end{exercise}
\begin{exercise}{456/7}
  \item [a]
  \begin{gather*}
    \text{Gegenwahrscheinlichkeit (keine Sechs werfen): } P(\neg E) = (\frac{5}{6})^5 \approx 0.40 \\
    P(E) = 1 - P(\neg E) = 1 - (\frac{5}{6})^5 \approx 0.60
  \end{gather*}
  \item [b]
  \begin{gather*}
    P = \frac{6}{6} \cdot \frac{5}{6} \cdot \frac{4}{6} \cdot \frac{3}{6} \cdot \frac{2}{6} = \approx 0.09
  \end{gather*}
  \item [c]
  \begin{gather*}
    P = (\frac{5}{6})^4 \cdot \frac{1}{6} \approx 0.08
  \end{gather*}
\end{exercise}
\begin{exercise}{456/9}
  \item [a]
  \begin{gather*}
    P = \frac{1}{13983816} \approx 7.15 \cdot 10^{-8}
  \end{gather*} \\\\
  Bemerkung: Ein Zufallsversuch heißt Laplace-Versuch, wenn alle Ergebnisse gleich wahrscheinlich sind.
  \begin{gather*}
    S = \{e_1, e_2, ..., e_n\} \qquad |S| = n \qquad P(e_i) = \frac{1}{n} \\
    E = \{e_1, ...\} \qquad |E| = k \qquad P(E) = \frac{k}{n}
  \end{gather*}
  $\Rightarrow$ Gleichverteilung
  \item [b]
  \begin{gather*}
    E = \{(1,2,3,4,5,6), (2,3,4,5,6,7), ..., (44,45,46,47,48,49)\} \\
    |E| = 44 \\
    P(E) = \frac{|E|}{|S|} = \frac{44}{13983816} \approx 3.15 \cdot 10^{-6}
  \end{gather*}
  \item [c]
  \begin{gather*}
    P = \frac{43}{49} \cdot \frac{42}{48} \cdot \frac{41}{47} \cdot \frac{40}{46} \cdot \frac{39}{45} \cdot \frac{38}{44} \approx 0.436
  \end{gather*}
\end{exercise}
\begin{exercise}{456/10}
  \item [a]
  \begin{gather*}
    \text{Anteil nach $10$ Tagen: } 0.85^{10} \approx 0.197 \\
    0.85^t = 0.5 \quad\Rightarrow\quad t = log_{0.85}(0.5) \approx 4.265
  \end{gather*}
  \item [b]
  \begin{gather*}
    (1 - p)^8 = 0.5 \quad\Rightarrow\quad p = 1 - \sqrt[8]{(0.5)} \approx 0.083
  \end{gather*}
\end{exercise}
\section{Abzählverfahren}
Bis auf Weiteres gilt: Die Zufallsversuche sind Laplace-Versuche. \\
Anzahl der Ergebnisse $N \quad\Rightarrow\quad P = \frac{1}{N} \quad\text{(Ergebniswahrscheinlichkeit)}$ \\\\
Ziehe $k$ mal aus einer Urne mit $n$ unterscheidbaren Kugeln:
\begin{itemize}
  \item mit Zurücklegen, unter Beachtung der Reihenfolge
  \begin{gather*}
    N = n^k
  \end{gather*}
  \item ohne Zurücklegen, unter Beachtung der Reihenfolge
  \begin{gather*}
    N = n \cdot (n - 1) \cdot (n - 2) \cdot ... \cdot \underbrace{(n - k + 1)}_{\mathclap{\text{übrige Kugeln vor dem $k$-ten Zug}}} = \frac{n!}{(n - k)!}
  \end{gather*}
  \item ohne Zurücklegen, ohne Beachtung der Reihenfolge \\\\
  Es werden jeweils $k!$ (Permutationen) Ergebnisse zu einem Ereignis zusammengefasst:
  \begin{gather*}
    N = \frac{n!}{(n - k)! \cdot k!} = \begin{pmatrix}n \\ k\end{pmatrix} \text{ (Binomialkoeffizient)}
  \end{gather*} \\
  Die Wahrscheinlichkeit für einen 6er im Lotto:
  \begin{gather*}
    p = \frac{1}{\begin{pmatrix}49 \\ 6\end{pmatrix}} \approx \frac{1}{14000000}
  \end{gather*}
  Die Wahrscheinlichkeit für einen 4er im Lotto (``Lottoproblem''):
  \begin{gather*}
    p = \frac{\begin{pmatrix}6 \\ 4\end{pmatrix} \cdot \begin{pmatrix}43 \\ 2\end{pmatrix}}{\begin{pmatrix}49 \\ 6\end{pmatrix}} = \frac{13545}{\begin{pmatrix}49 \\ 6\end{pmatrix}} \approx \frac{1}{1000}
  \end{gather*}
  \qquad (Wähle $4$ aus den $6$ Richtigen und $2$ aus den $43$ Falschen)
\end{itemize}
\begin{exercise}{460/1}
  \item [a]
  \begin{gather*}
    n = 2 \quad k = 6 \quad N = 2^6 = 64
  \end{gather*}
  \item [b]
  \begin{gather*}
    n = 4 \quad k = 4 \quad N = \frac{4!}{(4 - 4)!} = 4! = 24
  \end{gather*}
  \item [c]
  \begin{gather*}
    n = 3 \quad k = 5 \quad N = 3^5 = 243
  \end{gather*}
  \item [d]
  \begin{gather*}
    n = 10 \quad k = 2 \quad N = \begin{pmatrix}10 \\ 2\end{pmatrix} = 45
  \end{gather*}
\end{exercise}
\begin{exercise}{460/2}
  \item [a]
  \begin{gather*}
    n = 4 \quad k = 4 \quad N = 4^4 = 256
  \end{gather*}
  \item [b]
  \begin{gather*}
    p_1 = \frac{1}{2} \cdot \frac{1}{4} \cdot \frac{1}{8} \cdot \frac{1}{8} = \frac{1}{512} \\
    p_2 = \frac{4!}{p_1} = \frac{3}{64} \\
    p_3 = 1 - (\frac{1}{2})^4 = \frac{15}{16}
  \end{gather*}
\end{exercise}
\begin{exercise}{460/3}
  \item [a] Die Reihenfolge der Ziffern entspricht der Reihenfolge der Spiele.
  \item [b]
  \begin{gather*}
    p = (\frac{1}{3})^{11} = \frac{1}{177147} \approx 5.65 \cdot 10^{-6}
  \end{gather*}
  Annahme: Heimsieg, Gastsieg und Unentschieden sind jeweils gleich wahrscheinlich.
  \item [c]
  \begin{gather*}
    N = (\frac{2}{3})^{11} \cdot 3^{11} = 2^{11} = 2048
  \end{gather*}
\end{exercise}
\begin{exercise}{460/5}
  \begin{gather*}
    N = \frac{12!}{(12 - 3)!} = 1320
  \end{gather*}
\end{exercise}
\begin{exercise}{461/8}
  \item [a]
  \begin{gather*}
    N = 5! = 120
  \end{gather*}
  \item [b]
  \begin{gather*}
    p = \frac{1}{5}
  \end{gather*}
  \item [c]
  \begin{gather*}
    \frac{1}{5} \cdot \frac{1}{4} = \frac{1}{20}
  \end{gather*}
\end{exercise}
\begin{exercise}{461/11}
  \item [a]
  \begin{gather*}
    p = \frac{6}{49} \cdot \frac{5}{48} \cdot \frac{4}{47} \cdot \frac{3}{46} \cdot \frac{43}{45} \cdot \frac{42}{44} \approx 6.46 \cdot 10^{-5}
  \end{gather*}
  \item [b]
  \begin{gather*}
    rrrrff, rrrfrf, rrfrrf, rfrrrf, frrrrf, \\
    rrrffr, rrfrfr, rfrrfr, frrrfr, \\
    rrffrr, rfrfrr, frrfrr, \\
    rffrrr, frfrrr, \\
    ffrrrr
  \end{gather*}
  \item [c]
  \begin{gather*}
    p_4 \approx 15 \cdot 6.46 \cdot 10^{-5} \approx 9.69 \cdot 10^{-4}
  \end{gather*}
  \item [d]
  \begin{gather*}
    p_2 = \begin{pmatrix}6 \\ 2\end{pmatrix} \cdot (\frac{6}{49} \cdot \frac{5}{48} \cdot \frac{43}{47} \cdot \frac{42}{46} \cdot \frac{41}{45} \cdot \frac{40}{44}) \approx 0.13
  \end{gather*}
  \item [e]
  \begin{gather*}
    p_0 = \begin{pmatrix}6 \\ 0\end{pmatrix} \cdot (\frac{43}{49} \cdot \frac{42}{48} \cdot \frac{41}{47} \cdot \frac{40}{46} \cdot \frac{39}{45} \cdot \frac{38}{44}) \approx 0.44 \\
    p_1 = \begin{pmatrix}6 \\ 1\end{pmatrix} \cdot (\frac{6}{49} \cdot \frac{43}{48} \cdot \frac{42}{47} \cdot \frac{41}{46} \cdot \frac{40}{45} \cdot \frac{39}{44}) \approx 0.41 \\
    p_{\geq 3} = 1 - (p_0 + p_1 + p_2) \approx 1.86 \cdot 10^{-2} \\\\
    p_{50Sp.} = 1 - (p_0 + p_1 + p_2)^{50} \approx 0.61 \\
    p_{100Sp.} = 1 - (p_0 + p_1 + p_2)^{100} \approx 0.85 \\
    p_{1000Sp.} = 1 - (p_0 + p_1 + p_2)^{1000} \approx 1.00
  \end{gather*}
\end{exercise}
\begin{exercise}{461/10}
  \item [a]
  \begin{gather*}
    p = \begin{pmatrix}6 \\ 6\end{pmatrix} \cdot (\frac{6}{45} \cdot \frac{5}{44} \cdot \frac{4}{43} \cdot \frac{3}{42} \cdot \frac{2}{41} \cdot \frac{1}{40}) = \frac{1}{\begin{pmatrix}45 \\ 6\end{pmatrix}} \approx 1.23 \cdot 10^{-7}
  \end{gather*}
  \item [b]
  \begin{gather*}
    N = \begin{pmatrix}5 \\ 2\end{pmatrix} = 10
  \end{gather*}
  \item [c]
  \begin{gather*}
    N = \begin{pmatrix}1000 \\ 2\end{pmatrix} = 4.995 \cdot 10^{5}
  \end{gather*}
\end{exercise}
\subsubsection{AB Grundbegriffe der Wahrscheinlichkeitsrechnung}
\begin{exercise}{AB/12}
  \begin{gather*}
    p = \frac{6}{7} \cdot \frac{4}{5} \cdot \frac{2}{3} \approx 0.46
  \end{gather*}
\end{exercise}
\begin{exercise}{AB/17}
  \item [a]
  \begin{gather*}
    N = \begin{pmatrix}16 \\ 3\end{pmatrix} \cdot \begin{pmatrix}8 \\ 2\end{pmatrix} = 39200
  \end{gather*}
  \item [b]
  \begin{gather*}
    N = \begin{pmatrix}24 \\ 5\end{pmatrix} - \begin{pmatrix}16 \\ 5\end{pmatrix} = 38136
  \end{gather*}
\end{exercise}
\begin{exercise}{AB/20}
  \begin{gather*}
    p_{3r} = \frac{\begin{pmatrix}5 \\ 3\end{pmatrix} \cdot \begin{pmatrix}15 \\ 5\end{pmatrix}}{\begin{pmatrix}20 \\ 8\end{pmatrix}} \approx 0.2384 \\
    p_{\geq 4r} = \frac{\begin{pmatrix}5 \\ 4\end{pmatrix} \cdot \begin{pmatrix}15 \\ 4\end{pmatrix}}{\begin{pmatrix}20 \\ 8\end{pmatrix}} + \frac{\begin{pmatrix}5 \\ 5\end{pmatrix} \cdot \begin{pmatrix}15 \\ 3\end{pmatrix}}{\begin{pmatrix}20 \\ 8\end{pmatrix}} \approx 0.0578
  \end{gather*}
\end{exercise}
\begin{exercise}{AB/21}
  \begin{gather*}
    p_2 = \frac{\begin{pmatrix}10 \\ 2\end{pmatrix} \cdot \begin{pmatrix}90 \\ 3\end{pmatrix}}{\begin{pmatrix}100 \\ 5\end{pmatrix}} \approx 0.0702 \\\\
    p_{\geq 3} = p_3 + p_4 + p_5 \\
    \;= \frac{\begin{pmatrix}10 \\ 3\end{pmatrix} \cdot \begin{pmatrix}90 \\ 2\end{pmatrix} + \begin{pmatrix}10 \\ 4\end{pmatrix} \cdot \begin{pmatrix}90 \\ 1\end{pmatrix} + \begin{pmatrix}10 \\ 5\end{pmatrix} \cdot \begin{pmatrix}90 \\ 0\end{pmatrix}}{\begin{pmatrix}100 \\ 5\end{pmatrix}} \approx 0.0066
  \end{gather*}
\end{exercise}
